\documentclass[12pt, a4paper, titlepage]{article}
\usepackage[utf8]{inputenc}
\usepackage[a4paper]{geometry}
\usepackage[T1]{fontenc}
\usepackage[english, icelandic]{babel}
\usepackage{verbatim}
\usepackage{fancyhdr}
\usepackage{tabulary}
\usepackage{caption}
\usepackage[pdftex]{graphicx}
\usepackage{chngpage}
\usepackage{multirow}
\usepackage{hyperref}
\usepackage{cleveref}
\usepackage{amsmath}
\usepackage{amsfonts}
 

%%%%%%% SETUP %%%%%%%%
\newcommand{\HRule}{\rule{\linewidth}{0.5mm}}
\frenchspacing
\pagestyle{fancy}
\chead{\small \textsc{}}
\rhead{\small Artificial Intelligence\textsc{}}
\lhead{\small Háskólinn í Reykjavík \textsc{}}

%%%% END OF SETUP %%%%
\begin{document}
\newcommand{\maketitlepage}[1]
{
    \begin{titlepage}
    
    \begin{center}
        \includegraphics[width=0.55\textwidth]{default_white}\\[1.5cm]
        \HRule \\[0.4cm]
        \textsc{\huge Project Report}\\[0.4cm]
        \textsc{\large Artificial Intelligence}\\[0.2cm]
        \HRule \\[0.4cm]

        
        {\textsc{\large Solving Sudoku as a Constraint Problem}}\\[0.4cm]
        \textsc{\Large }\\[1.25cm]
        
        
        \textsc{
        Ingimar Örn Oddsson} \\
        \textsc{
        Johan Ejstrud} \\
        


        \vfill
        
        \textsc{}\\
    \end{center}
    
    \end{titlepage}
}

%%% Local Variables:
%%% mode: latex
%%% TeX-master: "sudoku_report"
%%% End:

\maketitlepage

\section*{Introduction}
The goal of our project was to solve a sudoku as a constraint satisfaction problem. Potentially we would also like to make an algorithm that can create sudokus with a unique solution. To make it more interesting we wanted to solve the problem more generally than the standard $9 \times 9$ case. We wanted to solve it for grid sizes of $n^2$, so 4, 9, 16, 25 etc.

\section*{Approach}
Because we wanted to solve the problem for all grid sizes, we could not just write out the constraints in verbose. Our initial thought was to find a way to of generating the constraints for the knowledge base given the size of the grid. However, if the sudoku is modeled with binary constraints one needs a lot of constraints. For example in the $4 \times 4$ grid cell (1,1) needs to be different from cell (1,2), (1,3), (1,4), (2,1), (2,1), (3,1) and (4,1). Since there are 16 cells there is a total of $\frac{7\cdot 16}{2} = 56$ constraints for the smallest puzzle. We divide by 2 because if we count 7 constraints for each cell we get each constraint twice. 

In general for at grid with size $d=n^2$ there are $2 \cdot (d-1) + (d-1)^2$ constraints for each cell. Hence the total number of constraints are $\frac{(2 \cdot (d-1) + (d-1)^2) \cdot d^2}{2}$. This increases quickly, for a $9 \times 9$ grid we get 810 constraints, $16 \times 16$ gives 4992 and $25 \times 25$ gives 20000 constraints. It seems apparent, that even if the constraints are generated automatically there are still way too many constraints for it to be a feasible solution.  

We then realized that for the sudoku, we do not actually need to explicitly module the constraints as we did in the lab project. Instead we update the domain of each variable using forward checking. Each time we assign a value, we simply remove that value from the domain of all cells in the same row, column or sub-grid as the active cell. Therefore the constraints are not explicitly written out, but are instead implicitly implemented in the rule we use to update the domains of each variable. If we do backtracking we have to be careful, because we cannot just use the ``reverse'' of the rule. When we backtrack, by removing a variable from an assignment, we have to check for each cell in the same row, column and sub-grid if the removed value is valid for this cell. This is quite costly, so instead for each state we save a list of the states which domain was affected by the most recently variable assignment. 

\section*{Conclusion}
We tested on $4 \times 4$, $9 \times 9$ and $16 \times 16$, and our script solved them all successfully. The running times for the different cases can be seen in \Cref{tab1}.    

\begin{table}[]
\centering
\begin{tabular}{c|c|c}
\textbf{Size }        &\textbf{ Difficulty }& \textbf{Time (ms)} \\ \hline
$4 \times 4$ & Hard       & 1         \\ \hline
$9 \times 9$ & Easy       & 8         \\ \hline
$9 \times 9$ & Evil       & 45        \\ \hline
$16 \times 16$ & Evil     & 80908     \\ 
\end{tabular}
\caption{Test results}
\label{tab1}
\end{table} 

\section*{Future Work}
We lost a group member in the middle of the project, and to reduce the work load, we decided not to make the sudoku creator. Our idea was to use our solving script to make a sudoku creator, that given a board size returns a random sudoku with a unique solution. 



\end{document}